% Options for packages loaded elsewhere
\PassOptionsToPackage{unicode}{hyperref}
\PassOptionsToPackage{hyphens}{url}
%
\documentclass[
]{article}
\usepackage{amsmath,amssymb}
\usepackage{lmodern}
\usepackage{iftex}
\ifPDFTeX
  \usepackage[T1]{fontenc}
  \usepackage[utf8]{inputenc}
  \usepackage{textcomp} % provide euro and other symbols
\else % if luatex or xetex
  \usepackage{unicode-math}
  \defaultfontfeatures{Scale=MatchLowercase}
  \defaultfontfeatures[\rmfamily]{Ligatures=TeX,Scale=1}
\fi
% Use upquote if available, for straight quotes in verbatim environments
\IfFileExists{upquote.sty}{\usepackage{upquote}}{}
\IfFileExists{microtype.sty}{% use microtype if available
  \usepackage[]{microtype}
  \UseMicrotypeSet[protrusion]{basicmath} % disable protrusion for tt fonts
}{}
\makeatletter
\@ifundefined{KOMAClassName}{% if non-KOMA class
  \IfFileExists{parskip.sty}{%
    \usepackage{parskip}
  }{% else
    \setlength{\parindent}{0pt}
    \setlength{\parskip}{6pt plus 2pt minus 1pt}}
}{% if KOMA class
  \KOMAoptions{parskip=half}}
\makeatother
\usepackage{xcolor}
\usepackage[margin=1in]{geometry}
\usepackage{color}
\usepackage{fancyvrb}
\newcommand{\VerbBar}{|}
\newcommand{\VERB}{\Verb[commandchars=\\\{\}]}
\DefineVerbatimEnvironment{Highlighting}{Verbatim}{commandchars=\\\{\}}
% Add ',fontsize=\small' for more characters per line
\usepackage{framed}
\definecolor{shadecolor}{RGB}{248,248,248}
\newenvironment{Shaded}{\begin{snugshade}}{\end{snugshade}}
\newcommand{\AlertTok}[1]{\textcolor[rgb]{0.94,0.16,0.16}{#1}}
\newcommand{\AnnotationTok}[1]{\textcolor[rgb]{0.56,0.35,0.01}{\textbf{\textit{#1}}}}
\newcommand{\AttributeTok}[1]{\textcolor[rgb]{0.77,0.63,0.00}{#1}}
\newcommand{\BaseNTok}[1]{\textcolor[rgb]{0.00,0.00,0.81}{#1}}
\newcommand{\BuiltInTok}[1]{#1}
\newcommand{\CharTok}[1]{\textcolor[rgb]{0.31,0.60,0.02}{#1}}
\newcommand{\CommentTok}[1]{\textcolor[rgb]{0.56,0.35,0.01}{\textit{#1}}}
\newcommand{\CommentVarTok}[1]{\textcolor[rgb]{0.56,0.35,0.01}{\textbf{\textit{#1}}}}
\newcommand{\ConstantTok}[1]{\textcolor[rgb]{0.00,0.00,0.00}{#1}}
\newcommand{\ControlFlowTok}[1]{\textcolor[rgb]{0.13,0.29,0.53}{\textbf{#1}}}
\newcommand{\DataTypeTok}[1]{\textcolor[rgb]{0.13,0.29,0.53}{#1}}
\newcommand{\DecValTok}[1]{\textcolor[rgb]{0.00,0.00,0.81}{#1}}
\newcommand{\DocumentationTok}[1]{\textcolor[rgb]{0.56,0.35,0.01}{\textbf{\textit{#1}}}}
\newcommand{\ErrorTok}[1]{\textcolor[rgb]{0.64,0.00,0.00}{\textbf{#1}}}
\newcommand{\ExtensionTok}[1]{#1}
\newcommand{\FloatTok}[1]{\textcolor[rgb]{0.00,0.00,0.81}{#1}}
\newcommand{\FunctionTok}[1]{\textcolor[rgb]{0.00,0.00,0.00}{#1}}
\newcommand{\ImportTok}[1]{#1}
\newcommand{\InformationTok}[1]{\textcolor[rgb]{0.56,0.35,0.01}{\textbf{\textit{#1}}}}
\newcommand{\KeywordTok}[1]{\textcolor[rgb]{0.13,0.29,0.53}{\textbf{#1}}}
\newcommand{\NormalTok}[1]{#1}
\newcommand{\OperatorTok}[1]{\textcolor[rgb]{0.81,0.36,0.00}{\textbf{#1}}}
\newcommand{\OtherTok}[1]{\textcolor[rgb]{0.56,0.35,0.01}{#1}}
\newcommand{\PreprocessorTok}[1]{\textcolor[rgb]{0.56,0.35,0.01}{\textit{#1}}}
\newcommand{\RegionMarkerTok}[1]{#1}
\newcommand{\SpecialCharTok}[1]{\textcolor[rgb]{0.00,0.00,0.00}{#1}}
\newcommand{\SpecialStringTok}[1]{\textcolor[rgb]{0.31,0.60,0.02}{#1}}
\newcommand{\StringTok}[1]{\textcolor[rgb]{0.31,0.60,0.02}{#1}}
\newcommand{\VariableTok}[1]{\textcolor[rgb]{0.00,0.00,0.00}{#1}}
\newcommand{\VerbatimStringTok}[1]{\textcolor[rgb]{0.31,0.60,0.02}{#1}}
\newcommand{\WarningTok}[1]{\textcolor[rgb]{0.56,0.35,0.01}{\textbf{\textit{#1}}}}
\usepackage{longtable,booktabs,array}
\usepackage{calc} % for calculating minipage widths
% Correct order of tables after \paragraph or \subparagraph
\usepackage{etoolbox}
\makeatletter
\patchcmd\longtable{\par}{\if@noskipsec\mbox{}\fi\par}{}{}
\makeatother
% Allow footnotes in longtable head/foot
\IfFileExists{footnotehyper.sty}{\usepackage{footnotehyper}}{\usepackage{footnote}}
\makesavenoteenv{longtable}
\usepackage{graphicx}
\makeatletter
\def\maxwidth{\ifdim\Gin@nat@width>\linewidth\linewidth\else\Gin@nat@width\fi}
\def\maxheight{\ifdim\Gin@nat@height>\textheight\textheight\else\Gin@nat@height\fi}
\makeatother
% Scale images if necessary, so that they will not overflow the page
% margins by default, and it is still possible to overwrite the defaults
% using explicit options in \includegraphics[width, height, ...]{}
\setkeys{Gin}{width=\maxwidth,height=\maxheight,keepaspectratio}
% Set default figure placement to htbp
\makeatletter
\def\fps@figure{htbp}
\makeatother
\usepackage[normalem]{ulem}
\setlength{\emergencystretch}{3em} % prevent overfull lines
\providecommand{\tightlist}{%
  \setlength{\itemsep}{0pt}\setlength{\parskip}{0pt}}
\setcounter{secnumdepth}{-\maxdimen} % remove section numbering
\usepackage{booktabs}
\usepackage{longtable}
\usepackage{array}
\usepackage{multirow}
\usepackage{wrapfig}
\usepackage{float}
\usepackage{colortbl}
\usepackage{pdflscape}
\usepackage{tabu}
\usepackage{threeparttable}
\usepackage{threeparttablex}
\usepackage[normalem]{ulem}
\usepackage{makecell}
\usepackage{xcolor}
\ifLuaTeX
  \usepackage{selnolig}  % disable illegal ligatures
\fi
\IfFileExists{bookmark.sty}{\usepackage{bookmark}}{\usepackage{hyperref}}
\IfFileExists{xurl.sty}{\usepackage{xurl}}{} % add URL line breaks if available
\urlstyle{same} % disable monospaced font for URLs
\hypersetup{
  pdftitle={Untitled},
  pdfauthor={Taiana Santos},
  hidelinks,
  pdfcreator={LaTeX via pandoc}}

\title{Untitled}
\author{Taiana Santos}
\date{2023-04-16}

\begin{document}
\maketitle

\hypertarget{introduuxe7uxe3o}{%
\subsection{Introdução}\label{introduuxe7uxe3o}}

O desenvolvimento do projeto foi para avaliação da Pós-Graduação MBA em
Data Science, disciplina Analise Exploratoria de Dados. Utilizamos a
linguagem de programação R, com as bibliotecas para criação de
relatorios ``rmarkdown'' e graficos interativos ``shiny''.

\hypertarget{banco-de-dados}{%
\subsection{Banco de Dados}\label{banco-de-dados}}

Para aplicar os conhecimentos, utilizamos os dados do Ranking de
Reclamações coletados no Banco Central do Brasil do 4º semestre de 2022.
Os dados divulgados nas listagens do ranking estão disponíveis para
download no site \url{https://www.bcb.gov.br/ranking/index.asp} .

O Ranking de Reclamações contribui para a transparência das ações do
Banco Central do Brasil, trazendo ao conhecimento da sociedade o perfil
das reclamações que foram processadas, analisadas e encerradas pelo BCB.

O Ranking é formado a partir das reclamações do público registradas nos
canais de atendimento do Banco Central (BC): internet, correspondência,
presencialmente ou telefone (145). As demandas são analisadas de forma
amostral pelo BC.

Participam do Ranking: bancos (comerciais, múltiplos, cooperativos, de
investimentos), sociedades de crédito, financiamento e investimento
(SCFI), instituições de pagamento e administradoras de consórcio.

As principais Reclamações registradas são irregularidades em operações e
serviços de cartões de crédito e em serviços de crédito.

\hypertarget{seleuxe7uxe3o-de-variuxe1veis}{%
\subsubsection{Seleção de
variáveis}\label{seleuxe7uxe3o-de-variuxe1veis}}

A função retorna 163 Instituições financeiras em 14 variáveis, sendo 5
variáveis caracteres e 9 variáveis numéricas, utilizaremos as variáveis
abaixo:

\begin{itemize}
\item
  Instituição financeira: bancos (comerciais, múltiplos, cooperativos,
  de investimentos), sociedades de crédito, financiamento e investimento
  (SCFI), instituições de pagamento e administradoras de consórcio.
\item
  Reclamações reguladas procedentes: casos em que houve sinal de
  descumprimento de lei ou norma pela instituição financeira no período
  indicado.
\item
  Total de reclamações: total de reclamações reguladas procedentes,
  reclamações reguladas outras e reclamações não reguladas.
\item
  Total de clientes : O número de clientes é formado pela base conjugada
  de dados das instituições no Cadastro de Clientes do Sistema
  Financeiro Nacional (CCS) e no Sistema de Informações de Crédito do
  Banco Central (SCR), em nº de CPFs e CNPJs distintos. Os clientes são
  contados uma única vez por instituição ou conglomerado financeiro.
\item
  Índice : \uline{(reclamações reguladas procedentes *} 1.000.000) /
  número de clientes.
\item
  Razão de reclamação: Criaremos a variável utilizando o total de
  (reclamações procedentes / total de reclamações) * 100 .

  As instituições são classificadas pela ordem decrescente do índice de
  reclamações, ou seja, da mais reclamada para a menos reclamada. É
  possível consultar a relação de instituições que tiveram menos de
  trinta reclamações procedentes apresentada em ordem alfabética do
  nome.
\end{itemize}

\hypertarget{ambiente-rstudio-e-bibliotecas-utilizadas}{%
\subsubsection{Ambiente RStudio e Bibliotecas
utilizadas}\label{ambiente-rstudio-e-bibliotecas-utilizadas}}

\includegraphics{https://raw.githubusercontent.com/taianasantos/Pos/main/print.jpg}

\begin{Shaded}
\begin{Highlighting}[]
\FunctionTok{library}\NormalTok{(corrplot)}
\FunctionTok{library}\NormalTok{(dlookr)}
\FunctionTok{library}\NormalTok{(dplyr)}
\FunctionTok{library}\NormalTok{(ggplot2)}
\FunctionTok{library}\NormalTok{(readxl)}
\FunctionTok{library}\NormalTok{(summarytools)}
\FunctionTok{library}\NormalTok{(tidyverse)}
\FunctionTok{library}\NormalTok{(knitr)}
\FunctionTok{library}\NormalTok{(stringr)}
\FunctionTok{library}\NormalTok{(writexl)}
\FunctionTok{library}\NormalTok{(rstatix)}
\FunctionTok{library}\NormalTok{(ggpubr)}
\FunctionTok{library}\NormalTok{(patchwork)}
\FunctionTok{library}\NormalTok{(psych)}
\FunctionTok{library}\NormalTok{(ggcorrplot)}
\FunctionTok{library}\NormalTok{(GGally)}
\FunctionTok{library}\NormalTok{(mice)}
\FunctionTok{library}\NormalTok{(skimr)}

\FunctionTok{library}\NormalTok{(rmarkdown)}
\CommentTok{\#install.packages("tinytex")}
\FunctionTok{library}\NormalTok{(tinytex)}
\end{Highlighting}
\end{Shaded}

\hypertarget{carregando-o-banco-de-dados}{%
\subsubsection{Carregando o Banco de
Dados}\label{carregando-o-banco-de-dados}}

--

Utilizamos o github para download do arquivo baixado do Banco Central e
a função read\_excel() para criar o Banco de dados Bancos3 com os dados
do arquivo Bancos4.xlsx.

\begin{Shaded}
\begin{Highlighting}[]
\FunctionTok{download.file}\NormalTok{( }\AttributeTok{url =} \StringTok{"https://github.com/taianasantos/analise/blob/main/Bancos.xlsx?raw=true"}\NormalTok{, }\AttributeTok{quiet =} \ConstantTok{TRUE}\NormalTok{,  }\AttributeTok{mode =} \StringTok{"wb"}\NormalTok{, }\AttributeTok{destfile =} \StringTok{"Bancos4.xlsx"}\NormalTok{ )}

\NormalTok{Bancos }\OtherTok{\textless{}{-}}\NormalTok{ readxl }\SpecialCharTok{::} \FunctionTok{read\_excel}\NormalTok{(}\StringTok{"Bancos4.xlsx"}\NormalTok{)}
\end{Highlighting}
\end{Shaded}

--

\begin{Shaded}
\begin{Highlighting}[]
\NormalTok{knitr }\SpecialCharTok{::} \FunctionTok{kable}\NormalTok{(}\FunctionTok{head}\NormalTok{(Bancos), }\AttributeTok{digits =} \DecValTok{1}\NormalTok{ ,}\StringTok{"pipe"}\NormalTok{)}
\end{Highlighting}
\end{Shaded}

\begin{longtable}[]{@{}
  >{\raggedright\arraybackslash}p{(\columnwidth - 10\tabcolsep) * \real{0.5398}}
  >{\raggedleft\arraybackslash}p{(\columnwidth - 10\tabcolsep) * \real{0.0619}}
  >{\raggedleft\arraybackslash}p{(\columnwidth - 10\tabcolsep) * \real{0.1062}}
  >{\raggedleft\arraybackslash}p{(\columnwidth - 10\tabcolsep) * \real{0.1593}}
  >{\raggedleft\arraybackslash}p{(\columnwidth - 10\tabcolsep) * \real{0.0796}}
  >{\raggedleft\arraybackslash}p{(\columnwidth - 10\tabcolsep) * \real{0.0531}}@{}}
\toprule()
\begin{minipage}[b]{\linewidth}\raggedright
Instituicao
\end{minipage} & \begin{minipage}[b]{\linewidth}\raggedleft
Indice
\end{minipage} & \begin{minipage}[b]{\linewidth}\raggedleft
Procedentes
\end{minipage} & \begin{minipage}[b]{\linewidth}\raggedleft
Total\_Reclamacoes
\end{minipage} & \begin{minipage}[b]{\linewidth}\raggedleft
Clientes
\end{minipage} & \begin{minipage}[b]{\linewidth}\raggedleft
Razao
\end{minipage} \\
\midrule()
\endhead
ACESSO SOLUÇÕES DE PAGAMENTO S.A. - INSTITUIÇÃO DE PAGAMENTO & 15.5 & 86
& 179 & 5538591 & 48.0 \\
AGIBANK (conglomerado) & 112.4 & 224 & 545 & 1992640 & 41.1 \\
AME DIGITAL BRASIL INSTITUICAO DE PAGAMENTO LTDA & 1.3 & 40 & 92 &
31101626 & 43.5 \\
BANCO CSF S.A. & 6.3 & 51 & 165 & 8070375 & 30.9 \\
BANCO DIGIMAIS S.A. & 170.3 & 56 & 141 & 328722 & 39.7 \\
BANCO DO ESTADO DO PARÁ S.A. & 33.7 & 34 & 65 & 1009445 & 52.3 \\
\bottomrule()
\end{longtable}

\hypertarget{identificando-os-tipos-de-cada-variuxe1vel}{%
\subsubsection{Identificando os tipos de cada
variável}\label{identificando-os-tipos-de-cada-variuxe1vel}}

--

Utilizamos a função diagnose() ,biblioteca dlookr, para identificar as
variáveis para análise.

A função diagnose() do pacote dlookr que retorna por variável qual o
tipo dela, contagem de valores faltantes, frequência de faltantes em
relação à base toda.

\begin{Shaded}
\begin{Highlighting}[]
\NormalTok{Bancos }\SpecialCharTok{\%\textgreater{}\%}\NormalTok{ dlookr}\SpecialCharTok{::}\FunctionTok{diagnose}\NormalTok{()}
\end{Highlighting}
\end{Shaded}

\begin{verbatim}
## # A tibble: 6 x 6
##   variables         types     missing_count missing_percent unique_count uniqu~1
##   <chr>             <chr>             <int>           <dbl>        <int>   <dbl>
## 1 Instituicao       character             0               0           42   1    
## 2 Indice            numeric               0               0           42   1    
## 3 Procedentes       numeric               0               0           39   0.929
## 4 Total_Reclamacoes numeric               0               0           42   1    
## 5 Clientes          numeric               0               0           42   1    
## 6 Razao             numeric               0               0           42   1    
## # ... with abbreviated variable name 1: unique_rate
\end{verbatim}

\hypertarget{anuxe1lise-descritiva-da-base-de-dados}{%
\section{Análise descritiva da Base de
dados}\label{anuxe1lise-descritiva-da-base-de-dados}}

Para realizar essa análise, podemos utilizar a função descr do pacote
summarytools e realizar a leitura desses dados. Analisar a centralidade
dos dados, dipersão, assimetria e se há presença de outliers.

\begin{Shaded}
\begin{Highlighting}[]
\NormalTok{Bancos  }\SpecialCharTok{\%\textgreater{}\%}\NormalTok{ summarytools}\SpecialCharTok{::}\FunctionTok{descr}\NormalTok{()}
\end{Highlighting}
\end{Shaded}

\begin{verbatim}
## Non-numerical variable(s) ignored: Instituicao
\end{verbatim}

\begin{verbatim}
## Descriptive Statistics  
## Bancos  
## N: 42  
## 
##                         Clientes   Indice   Procedentes    Razao   Total_Reclamacoes
## ----------------- -------------- -------- ------------- -------- -------------------
##              Mean    21821567.14    69.91        418.57    38.43             1294.95
##           Std.Dev    33082609.72   124.29        539.46    13.13             1848.89
##               Min      179473.00     1.28         30.00     8.62               65.00
##                Q1     1992640.00    11.75         53.00    28.34              165.00
##            Median     6800159.50    23.66        185.00    39.26              462.00
##                Q3    23219194.00    63.88        652.00    48.04             1711.00
##               Max   148128348.00   638.66       2225.00    70.66             6878.00
##               MAD     8675307.14    23.80        217.94    15.53              485.55
##               IQR    20938345.00    50.85        587.75    19.46             1541.00
##                CV           1.52     1.78          1.29     0.34                1.43
##          Skewness           2.08     3.04          1.73    -0.01                1.86
##       SE.Skewness           0.37     0.37          0.37     0.37                0.37
##          Kurtosis           3.95     9.50          2.22    -0.41                2.35
##           N.Valid          42.00    42.00         42.00    42.00               42.00
##         Pct.Valid         100.00   100.00        100.00   100.00              100.00
\end{verbatim}

\begin{Shaded}
\begin{Highlighting}[]
\FunctionTok{view}\NormalTok{(}\FunctionTok{descr}\NormalTok{(Bancos))}
\end{Highlighting}
\end{Shaded}

As variaveis Clientes, Indice, Procedentes e Total\_Reclamacoes
apresentam presenças de outliers, influenciando muito o valor da média
que apresenta valores maiores que a Mediana. Com o valor de desvio
padrão mais alto indica maior dispersão nos dados, geralmente longe da
média.

\hypertarget{distribuiuxe7uxe3o-normal}{%
\section{Distribuição Normal}\label{distribuiuxe7uxe3o-normal}}

A distribuição normal é uma das mais utilizadas em estatística sendo
exigida em muitas situações pelas suas boas propriedades matemáticas e
por isso é tão utilizada em probabilidade e estatística, como no teste t
e na ANOVA.

Para verificar a normalidade da distribuição vamos apresentar três
ferramentas que devem ser utilizadas em conjunto:

\begin{enumerate}
\def\labelenumi{\arabic{enumi}.}
\item
  Histograma: É uma distribuição contínua, onde a distribuição de
  frequências de uma variável quantitativa apresenta a forma de sino e é
  simétrica em relação a sua média.

  Algumas caracteristicas:

  É simétrica em torno da média;

  A área sob a curva corresponde à proporção 1 ou 100\%;

  As medidas de tendência central (média, mediana e moda) apresentam o
  mesmo valor;

  Os extremos da curva tendem ao infinito em ambas as direções e,
  teoricamente, jamais tocam o eixo x;
\end{enumerate}

\begin{Shaded}
\begin{Highlighting}[]
  \FunctionTok{set.seed}\NormalTok{(}\DecValTok{1}\NormalTok{)}
\NormalTok{    dds.normal }\OtherTok{\textless{}{-}} \FunctionTok{rnorm}\NormalTok{(}\DecValTok{100}\NormalTok{,}\DecValTok{2}\NormalTok{,}\DecValTok{1}\NormalTok{) }\CommentTok{\#Normal}
\NormalTok{    dds.exp }\OtherTok{\textless{}{-}} \FunctionTok{rexp}\NormalTok{(}\DecValTok{100}\NormalTok{)  }\CommentTok{\# Exponencial }
    \CommentTok{\# Histogramas}

    \FunctionTok{par}\NormalTok{(}\AttributeTok{mfrow=}\FunctionTok{c}\NormalTok{(}\DecValTok{1}\NormalTok{,}\DecValTok{2}\NormalTok{))}

    \FunctionTok{hist}\NormalTok{(dds.normal, }
         \AttributeTok{col =} \StringTok{"\#D9D9D9"}\NormalTok{,}
         \AttributeTok{freq =}\NormalTok{ F, }
         \AttributeTok{main =} \StringTok{"Histograma Normal"}\NormalTok{, }\AttributeTok{xlab =} \StringTok{"Distribuição Normal"}\NormalTok{)}
    \CommentTok{\#curve(dnorm(x, mean = mean(dds.normal), sd = sd(dds.normal)), add = T)}

    \FunctionTok{hist}\NormalTok{(dds.exp,}
         \AttributeTok{col =} \StringTok{"\#38761D"}\NormalTok{,}
         \AttributeTok{freq =}\NormalTok{ F, }
         \AttributeTok{main =} \StringTok{"Histograma não Normal"}\NormalTok{, }\AttributeTok{xlab =} \StringTok{"Distribuição Exponencial"}\NormalTok{)}
\end{Highlighting}
\end{Shaded}

\includegraphics{Projeto2_files/figure-latex/unnamed-chunk-6-1.pdf}

\begin{enumerate}
\def\labelenumi{\arabic{enumi}.}
\setcounter{enumi}{1}
\item
  QQ-plot : grafico utilizado para comprar quartil por quartil entre uma
  distribuição normal padrão e a distribuição que se visa comparar após
  ser padronizada. Caso o Q-Q Plot tenha comportamento comportamento de
  uma reta perfeitamente ascendente, é a sinalização de que a
  distribuição que está estudando se aproxima de uma distribuição
  normal. No entanto, caso perceba muita variação, sobretudo nos
  extremos, terá a sinalização de que a distribuição estudada não é
  normal.

\begin{Shaded}
\begin{Highlighting}[]
\FunctionTok{par}\NormalTok{(}\AttributeTok{mfrow=}\FunctionTok{c}\NormalTok{(}\DecValTok{1}\NormalTok{,}\DecValTok{2}\NormalTok{))}

\CommentTok{\# Se os pontos se concentrarem em torno de uma linha reta, \# então temos indícios de que a distribuição é normal. }
\FunctionTok{qqnorm}\NormalTok{(dds.normal, }\AttributeTok{main =} \StringTok{"Q{-}Q Plot Normal"}\NormalTok{, }\AttributeTok{xlab =} \StringTok{"Distribuição Normal"}\NormalTok{) }
\FunctionTok{qqline}\NormalTok{(dds.normal, }\AttributeTok{col =} \StringTok{"red"}\NormalTok{) }

\FunctionTok{qqnorm}\NormalTok{(dds.exp, }\AttributeTok{main =} \StringTok{" Q{-}Q Plot Não Normal"}\NormalTok{, }\AttributeTok{xlab =} \StringTok{"Distribuição Exponencial"}\NormalTok{) }
\FunctionTok{qqline}\NormalTok{(dds.exp, }\AttributeTok{col =} \StringTok{"red"}\NormalTok{) }
\end{Highlighting}
\end{Shaded}

  \includegraphics{Projeto2_files/figure-latex/qq-plot + linha de referência-1.pdf}
\item
  Teste Shapiro-Wilk;

  Se p- value \textgreater{} 0,05, então a distribuição dos dados
  fornecidos não difere significativamente da distribuição normal. Se
  p-value \textless{} 0.05, então está rejeitada a hipótese de
  distribuição normal.
\end{enumerate}

\begin{Shaded}
\begin{Highlighting}[]
\NormalTok{rstatix}\SpecialCharTok{::} \FunctionTok{shapiro\_test}\NormalTok{(dds.normal) }\CommentTok{\# normal }
\end{Highlighting}
\end{Shaded}

\begin{verbatim}
## # A tibble: 1 x 3
##   variable   statistic p.value
##   <chr>          <dbl>   <dbl>
## 1 dds.normal     0.996   0.988
\end{verbatim}

\begin{Shaded}
\begin{Highlighting}[]
\NormalTok{rstatix}\SpecialCharTok{::} \FunctionTok{shapiro\_test}\NormalTok{(dds.exp) }\CommentTok{\# não normal}
\end{Highlighting}
\end{Shaded}

\begin{verbatim}
## # A tibble: 1 x 3
##   variable statistic    p.value
##   <chr>        <dbl>      <dbl>
## 1 dds.exp      0.915 0.00000776
\end{verbatim}

\hypertarget{normalidade-das-variuxe1veis}{%
\section{Normalidade das variáveis}\label{normalidade-das-variuxe1veis}}

Utilizando a função pairs.panels e com a analise rapida dos histogramas,
possivelmente a unica variavel com distribuição normal é a razão.

\begin{Shaded}
\begin{Highlighting}[]
\NormalTok{psych}\SpecialCharTok{::} \FunctionTok{pairs.panels}\NormalTok{(Bancos[, }\SpecialCharTok{{-}}\DecValTok{1}\NormalTok{], }
             \AttributeTok{method =} \StringTok{"pearson"}\NormalTok{, }\CommentTok{\# correlation method}
             \AttributeTok{hist.col =} \StringTok{"lightblue"}\NormalTok{,}
            \AttributeTok{density =} \ConstantTok{FALSE}\NormalTok{,  }\CommentTok{\# show density plots}
             \AttributeTok{ellipses =} \ConstantTok{FALSE} \CommentTok{\# show correlation ellipses}
\NormalTok{             )}
\end{Highlighting}
\end{Shaded}

\includegraphics{Projeto2_files/figure-latex/unnamed-chunk-7-1.pdf}

\hypertarget{nuxfamero-de-bins}{%
\subsection{Número de Bins}\label{nuxfamero-de-bins}}

A escolha do número de bins do histograma é uma escolha arbitrária e
está diretamente ligada à amplitude que se deseja definir como base de
cada barra contígua, para o caso onde se trabalha com amplitudes iguais
para todos intervalos.

Existem equações modeladas para sugerir amplitudes dos bins de modo a
facilitar a decisão do analista, das quais destacamos duas:

A regra de Freedman-Diaconis leva em consideração o intervalo
interquartil dos dados, que impede que outliers tenham infuência na
definição da amplitude do intervalo de bin.

A regra de Sturge leva em consideração a dispersão da distribuição para
definir a amplitude. Em geral, a regra de Sturge é mais recomendada
quando a distribuição dos dados se aproximará de uma distribuiçao
normal.

\begin{Shaded}
\begin{Highlighting}[]
\NormalTok{fd }\OtherTok{\textless{}{-}} \ControlFlowTok{function}\NormalTok{(x) \{}
\NormalTok{  n }\OtherTok{\textless{}{-}}\FunctionTok{length}\NormalTok{(x)}
  \FunctionTok{return}\NormalTok{((}\DecValTok{2}\SpecialCharTok{*}\FunctionTok{IQR}\NormalTok{(x))}\SpecialCharTok{/}\NormalTok{n}\SpecialCharTok{\^{}}\NormalTok{(}\DecValTok{1}\SpecialCharTok{/}\DecValTok{3}\NormalTok{))}
\NormalTok{\}}
\NormalTok{sr }\OtherTok{\textless{}{-}} \ControlFlowTok{function}\NormalTok{(x) \{}
\NormalTok{  n }\OtherTok{\textless{}{-}}\FunctionTok{length}\NormalTok{(x)}
  \FunctionTok{return}\NormalTok{((}\FloatTok{3.49}\SpecialCharTok{*}\FunctionTok{sd}\NormalTok{(x))}\SpecialCharTok{/}\NormalTok{n}\SpecialCharTok{\^{}}\NormalTok{(}\DecValTok{1}\SpecialCharTok{/}\DecValTok{3}\NormalTok{))}
\NormalTok{  \}}
\end{Highlighting}
\end{Shaded}

\hypertarget{variavel-indice}{%
\section{Variavel Indice}\label{variavel-indice}}

\hypertarget{distribuiuxe7uxe3o-nuxe3o-normal}{%
\subsection{Distribuição Não
Normal}\label{distribuiuxe7uxe3o-nuxe3o-normal}}

\hypertarget{histograma}{%
\subsubsection{Histograma}\label{histograma}}

\begin{Shaded}
\begin{Highlighting}[]
\NormalTok{g1 }\OtherTok{\textless{}{-}}\NormalTok{ Bancos }\SpecialCharTok{\%\textgreater{}\%}\NormalTok{ dplyr}\SpecialCharTok{::}\FunctionTok{select}\NormalTok{(Indice) }\SpecialCharTok{\%\textgreater{}\%} \FunctionTok{ggplot}\NormalTok{(}\FunctionTok{aes}\NormalTok{(}\AttributeTok{x=}\NormalTok{Indice))}\SpecialCharTok{+}\FunctionTok{geom\_histogram}\NormalTok{(}\FunctionTok{aes}\NormalTok{(}\AttributeTok{y =} \FunctionTok{after\_stat}\NormalTok{(density)) , }\AttributeTok{binwidth=}\NormalTok{fd, }\AttributeTok{fill =} \StringTok{\textquotesingle{}lightblue\textquotesingle{}}\NormalTok{) }\SpecialCharTok{+} \FunctionTok{xlab}\NormalTok{(}\StringTok{\textquotesingle{}Indice\textquotesingle{}}\NormalTok{) }\SpecialCharTok{+} \FunctionTok{ylab}\NormalTok{(}\StringTok{\textquotesingle{}Densidade de Frequencia\textquotesingle{}}\NormalTok{) }\SpecialCharTok{+} \FunctionTok{labs}\NormalTok{(}\AttributeTok{title =} \StringTok{"Distribuicao dos dados Indice aproximada por Histograma"}\NormalTok{, }\AttributeTok{subtitle =} \StringTok{"Binarizacao pela Regra de FD"}\NormalTok{) }\SpecialCharTok{+} \FunctionTok{geom\_vline}\NormalTok{(}\AttributeTok{xintercept=}\FunctionTok{c}\NormalTok{(}\FunctionTok{median}\NormalTok{(Bancos}\SpecialCharTok{$}\NormalTok{Indice), }\FunctionTok{mean}\NormalTok{(Bancos}\SpecialCharTok{$}\NormalTok{Indice))) }\SpecialCharTok{+} \FunctionTok{annotate}\NormalTok{(}\StringTok{"text"}\NormalTok{, }\AttributeTok{x=}\FunctionTok{median}\NormalTok{(Bancos}\SpecialCharTok{$}\NormalTok{Indice) }\SpecialCharTok{+} \FloatTok{0.3}\NormalTok{, }\AttributeTok{y=}\FloatTok{0.02}\NormalTok{, }\AttributeTok{label=}\StringTok{"Mediana"}\NormalTok{, }\AttributeTok{angle=}\DecValTok{90}\NormalTok{) }\SpecialCharTok{+} \FunctionTok{annotate}\NormalTok{(}\StringTok{"text"}\NormalTok{, }\AttributeTok{x=}\FunctionTok{mean}\NormalTok{(Bancos}\SpecialCharTok{$}\NormalTok{Indice) }\SpecialCharTok{+} \FloatTok{0.3}\NormalTok{, }\AttributeTok{y=}\FloatTok{0.02}\NormalTok{, }\AttributeTok{label=}\StringTok{"Media"}\NormalTok{, }\AttributeTok{angle=}\DecValTok{90}\NormalTok{) }\SpecialCharTok{+} \FunctionTok{geom\_density}\NormalTok{(}\AttributeTok{linetype =} \DecValTok{2}\NormalTok{) }\SpecialCharTok{+} \FunctionTok{theme\_classic}\NormalTok{()}


\NormalTok{g2 }\OtherTok{\textless{}{-}}\NormalTok{ Bancos }\SpecialCharTok{\%\textgreater{}\%}\NormalTok{ dplyr}\SpecialCharTok{::}\FunctionTok{select}\NormalTok{(Indice) }\SpecialCharTok{\%\textgreater{}\%} \FunctionTok{ggplot}\NormalTok{(}\FunctionTok{aes}\NormalTok{(}\AttributeTok{x=}\NormalTok{Indice))}\SpecialCharTok{+}\FunctionTok{geom\_histogram}\NormalTok{(}\FunctionTok{aes}\NormalTok{(}\AttributeTok{y =} \FunctionTok{after\_stat}\NormalTok{(density)) , }\AttributeTok{binwidth=}\NormalTok{sr, }\AttributeTok{fill =} \StringTok{\textquotesingle{}lightblue\textquotesingle{}}\NormalTok{) }\SpecialCharTok{+} \FunctionTok{xlab}\NormalTok{(}\StringTok{\textquotesingle{}Indice\textquotesingle{}}\NormalTok{) }\SpecialCharTok{+} \FunctionTok{ylab}\NormalTok{(}\StringTok{\textquotesingle{}Densidade de Frequencia\textquotesingle{}}\NormalTok{) }\SpecialCharTok{+} \FunctionTok{labs}\NormalTok{(}\AttributeTok{title =} \StringTok{"Distribuicao dos dados de Indice aproximada por Histograma"}\NormalTok{, }\AttributeTok{subtitle =} \StringTok{"Binarizacao pela Regra de SR"}\NormalTok{) }\SpecialCharTok{+} \FunctionTok{geom\_vline}\NormalTok{(}\AttributeTok{xintercept=}\FunctionTok{c}\NormalTok{(}\FunctionTok{median}\NormalTok{(Bancos}\SpecialCharTok{$}\NormalTok{Indice), }\FunctionTok{mean}\NormalTok{(Bancos}\SpecialCharTok{$}\NormalTok{Indice))) }\SpecialCharTok{+} \FunctionTok{annotate}\NormalTok{(}\StringTok{"text"}\NormalTok{, }\AttributeTok{x=}\FunctionTok{median}\NormalTok{(Bancos}\SpecialCharTok{$}\NormalTok{Indice) }\SpecialCharTok{+} \FloatTok{0.3}\NormalTok{, }\AttributeTok{y=}\FloatTok{0.02}\NormalTok{, }\AttributeTok{label=}\StringTok{"Mediana"}\NormalTok{, }\AttributeTok{angle=}\DecValTok{90}\NormalTok{) }\SpecialCharTok{+} \FunctionTok{annotate}\NormalTok{(}\StringTok{"text"}\NormalTok{, }\AttributeTok{x=}\FunctionTok{mean}\NormalTok{(Bancos}\SpecialCharTok{$}\NormalTok{Indice) }\SpecialCharTok{+} \FloatTok{0.3}\NormalTok{, }\AttributeTok{y=}\FloatTok{0.02}\NormalTok{, }\AttributeTok{label=}\StringTok{"Media"}\NormalTok{, }\AttributeTok{angle=}\DecValTok{90}\NormalTok{) }\SpecialCharTok{+} \FunctionTok{geom\_density}\NormalTok{(}\AttributeTok{linetype =} \DecValTok{2}\NormalTok{) }\SpecialCharTok{+} \FunctionTok{theme\_classic}\NormalTok{()}

\NormalTok{g1 }\SpecialCharTok{+}\NormalTok{ g2}
\end{Highlighting}
\end{Shaded}

\includegraphics{Projeto2_files/figure-latex/unnamed-chunk-9-1.pdf}

\hypertarget{qq-plot}{%
\subsubsection{Qq Plot}\label{qq-plot}}

Grafico Qqplot

\begin{Shaded}
\begin{Highlighting}[]
\NormalTok{ggpubr}\SpecialCharTok{::}\FunctionTok{ggqqplot}\NormalTok{(Bancos}\SpecialCharTok{$}\NormalTok{Indice)  }\SpecialCharTok{+}
 \FunctionTok{labs}\NormalTok{(}\AttributeTok{title=}\StringTok{"QQ plot Variavel Indice"}\NormalTok{,}
       \AttributeTok{y =} \StringTok{"Indice"}\NormalTok{)}
\end{Highlighting}
\end{Shaded}

\includegraphics{Projeto2_files/figure-latex/unnamed-chunk-10-1.pdf}

\hypertarget{teste-shapiro-wilk}{%
\subsubsection{Teste Shapiro-Wilk}\label{teste-shapiro-wilk}}

\begin{Shaded}
\begin{Highlighting}[]
\NormalTok{rstatix}\SpecialCharTok{::}\FunctionTok{shapiro\_test}\NormalTok{(Bancos}\SpecialCharTok{$}\NormalTok{Indice)}
\end{Highlighting}
\end{Shaded}

\begin{verbatim}
## # A tibble: 1 x 3
##   variable      statistic  p.value
##   <chr>             <dbl>    <dbl>
## 1 Bancos$Indice     0.544 3.12e-10
\end{verbatim}

\hypertarget{variavel-procedentes}{%
\section{Variavel Procedentes}\label{variavel-procedentes}}

\hypertarget{resultado-distribuiuxe7uxe3o-nuxe3o-normal}{%
\subsection{Resultado: Distribuição Não
Normal}\label{resultado-distribuiuxe7uxe3o-nuxe3o-normal}}

\hypertarget{histograma-1}{%
\subsubsection{Histograma}\label{histograma-1}}

\begin{Shaded}
\begin{Highlighting}[]
\NormalTok{g1 }\OtherTok{\textless{}{-}}\NormalTok{ Bancos }\SpecialCharTok{\%\textgreater{}\%}\NormalTok{ dplyr}\SpecialCharTok{::}\FunctionTok{select}\NormalTok{(Procedentes) }\SpecialCharTok{\%\textgreater{}\%} \FunctionTok{ggplot}\NormalTok{(}\FunctionTok{aes}\NormalTok{(}\AttributeTok{x=}\NormalTok{Procedentes))}\SpecialCharTok{+}\FunctionTok{geom\_histogram}\NormalTok{(}\FunctionTok{aes}\NormalTok{(}\AttributeTok{y =} \FunctionTok{after\_stat}\NormalTok{(density)) , }\AttributeTok{binwidth=}\NormalTok{fd, }\AttributeTok{fill =} \StringTok{\textquotesingle{}lightblue\textquotesingle{}}\NormalTok{) }\SpecialCharTok{+} \FunctionTok{xlab}\NormalTok{(}\StringTok{\textquotesingle{}Procedentes\textquotesingle{}}\NormalTok{) }\SpecialCharTok{+} \FunctionTok{ylab}\NormalTok{(}\StringTok{\textquotesingle{}Densidade de Frequencia\textquotesingle{}}\NormalTok{) }\SpecialCharTok{+} \FunctionTok{labs}\NormalTok{(}\AttributeTok{title =} \StringTok{"Distribuicao dos dados Indice aproximada por Histograma"}\NormalTok{, }\AttributeTok{subtitle =} \StringTok{"Binarizacao pela Regra de FD"}\NormalTok{) }\SpecialCharTok{+} \FunctionTok{geom\_vline}\NormalTok{(}\AttributeTok{xintercept=}\FunctionTok{c}\NormalTok{(}\FunctionTok{median}\NormalTok{(Bancos}\SpecialCharTok{$}\NormalTok{Procedentes), }\FunctionTok{mean}\NormalTok{(Bancos}\SpecialCharTok{$}\NormalTok{Procedentes))) }\SpecialCharTok{+} \FunctionTok{annotate}\NormalTok{(}\StringTok{"text"}\NormalTok{, }\AttributeTok{x=}\FunctionTok{median}\NormalTok{(Bancos}\SpecialCharTok{$}\NormalTok{Procedentes) }\SpecialCharTok{+} \FloatTok{0.3}\NormalTok{, }\AttributeTok{y=}\FloatTok{0.02}\NormalTok{, }\AttributeTok{label=}\StringTok{"Mediana"}\NormalTok{, }\AttributeTok{angle=}\DecValTok{90}\NormalTok{) }\SpecialCharTok{+} \FunctionTok{annotate}\NormalTok{(}\StringTok{"text"}\NormalTok{, }\AttributeTok{x=}\FunctionTok{mean}\NormalTok{(Bancos}\SpecialCharTok{$}\NormalTok{Procedentes) }\SpecialCharTok{+} \FloatTok{0.3}\NormalTok{, }\AttributeTok{y=}\FloatTok{0.02}\NormalTok{, }\AttributeTok{label=}\StringTok{"Media"}\NormalTok{, }\AttributeTok{angle=}\DecValTok{90}\NormalTok{) }\SpecialCharTok{+} \FunctionTok{geom\_density}\NormalTok{(}\AttributeTok{linetype =} \DecValTok{2}\NormalTok{) }\SpecialCharTok{+} \FunctionTok{theme\_classic}\NormalTok{()}


\NormalTok{g2 }\OtherTok{\textless{}{-}}\NormalTok{ Bancos }\SpecialCharTok{\%\textgreater{}\%}\NormalTok{ dplyr}\SpecialCharTok{::}\FunctionTok{select}\NormalTok{(Procedentes) }\SpecialCharTok{\%\textgreater{}\%} \FunctionTok{ggplot}\NormalTok{(}\FunctionTok{aes}\NormalTok{(}\AttributeTok{x=}\NormalTok{Procedentes))}\SpecialCharTok{+}\FunctionTok{geom\_histogram}\NormalTok{(}\FunctionTok{aes}\NormalTok{(}\AttributeTok{y =} \FunctionTok{after\_stat}\NormalTok{(density)) , }\AttributeTok{binwidth=}\NormalTok{sr, }\AttributeTok{fill =} \StringTok{\textquotesingle{}lightblue\textquotesingle{}}\NormalTok{) }\SpecialCharTok{+} \FunctionTok{xlab}\NormalTok{(}\StringTok{\textquotesingle{}Procedentes\textquotesingle{}}\NormalTok{) }\SpecialCharTok{+} \FunctionTok{ylab}\NormalTok{(}\StringTok{\textquotesingle{}Densidade de Frequencia\textquotesingle{}}\NormalTok{) }\SpecialCharTok{+} \FunctionTok{labs}\NormalTok{(}\AttributeTok{title =} \StringTok{"Distribuicao dos dados de Indice aproximada por Histograma"}\NormalTok{, }\AttributeTok{subtitle =} \StringTok{"Binarizacao pela Regra de SR"}\NormalTok{) }\SpecialCharTok{+} \FunctionTok{geom\_vline}\NormalTok{(}\AttributeTok{xintercept=}\FunctionTok{c}\NormalTok{(}\FunctionTok{median}\NormalTok{(Bancos}\SpecialCharTok{$}\NormalTok{Procedentes), }\FunctionTok{mean}\NormalTok{(Bancos}\SpecialCharTok{$}\NormalTok{Procedentes))) }\SpecialCharTok{+} \FunctionTok{annotate}\NormalTok{(}\StringTok{"text"}\NormalTok{, }\AttributeTok{x=}\FunctionTok{median}\NormalTok{(Bancos}\SpecialCharTok{$}\NormalTok{Procedentes) }\SpecialCharTok{+} \FloatTok{0.3}\NormalTok{, }\AttributeTok{y=}\FloatTok{0.02}\NormalTok{, }\AttributeTok{label=}\StringTok{"Mediana"}\NormalTok{, }\AttributeTok{angle=}\DecValTok{90}\NormalTok{) }\SpecialCharTok{+} \FunctionTok{annotate}\NormalTok{(}\StringTok{"text"}\NormalTok{, }\AttributeTok{x=}\FunctionTok{mean}\NormalTok{(Bancos}\SpecialCharTok{$}\NormalTok{Procedentes) }\SpecialCharTok{+} \FloatTok{0.3}\NormalTok{, }\AttributeTok{y=}\FloatTok{0.02}\NormalTok{, }\AttributeTok{label=}\StringTok{"Media"}\NormalTok{, }\AttributeTok{angle=}\DecValTok{90}\NormalTok{) }\SpecialCharTok{+} \FunctionTok{geom\_density}\NormalTok{(}\AttributeTok{linetype =} \DecValTok{2}\NormalTok{) }\SpecialCharTok{+} \FunctionTok{theme\_classic}\NormalTok{()}

\NormalTok{g1 }\SpecialCharTok{+}\NormalTok{ g2}
\end{Highlighting}
\end{Shaded}

\includegraphics{Projeto2_files/figure-latex/unnamed-chunk-12-1.pdf}

\hypertarget{qq-plot-1}{%
\subsubsection{Qq Plot}\label{qq-plot-1}}

Grafico Qqplot

\begin{Shaded}
\begin{Highlighting}[]
\NormalTok{ggpubr}\SpecialCharTok{::}\FunctionTok{ggqqplot}\NormalTok{(Bancos}\SpecialCharTok{$}\NormalTok{Procedentes)  }\SpecialCharTok{+}
 \FunctionTok{labs}\NormalTok{(}\AttributeTok{title=}\StringTok{"QQ plot Variavel Procedentes"}\NormalTok{,}
       \AttributeTok{y =} \StringTok{"Indice"}\NormalTok{)}
\end{Highlighting}
\end{Shaded}

\includegraphics{Projeto2_files/figure-latex/unnamed-chunk-13-1.pdf}

\hypertarget{teste-shapiro-wilk-1}{%
\subsubsection{Teste Shapiro-Wilk}\label{teste-shapiro-wilk-1}}

\begin{Shaded}
\begin{Highlighting}[]
\NormalTok{rstatix}\SpecialCharTok{::}\FunctionTok{shapiro\_test}\NormalTok{(Bancos}\SpecialCharTok{$}\NormalTok{Procedentes)}
\end{Highlighting}
\end{Shaded}

\begin{verbatim}
## # A tibble: 1 x 3
##   variable           statistic     p.value
##   <chr>                  <dbl>       <dbl>
## 1 Bancos$Procedentes     0.731 0.000000196
\end{verbatim}

\hypertarget{variavel-total_reclamacoes}{%
\section{Variavel Total\_Reclamacoes}\label{variavel-total_reclamacoes}}

\hypertarget{distribuiuxe7uxe3o-nuxe3o-normal-1}{%
\subsection{Distribuição Não
Normal}\label{distribuiuxe7uxe3o-nuxe3o-normal-1}}

\hypertarget{histograma-2}{%
\subsubsection{Histograma}\label{histograma-2}}

\begin{Shaded}
\begin{Highlighting}[]
\NormalTok{g1 }\OtherTok{\textless{}{-}}\NormalTok{ Bancos }\SpecialCharTok{\%\textgreater{}\%}\NormalTok{ dplyr}\SpecialCharTok{::}\FunctionTok{select}\NormalTok{(Total\_Reclamacoes ) }\SpecialCharTok{\%\textgreater{}\%} \FunctionTok{ggplot}\NormalTok{(}\FunctionTok{aes}\NormalTok{(}\AttributeTok{x=}\NormalTok{Total\_Reclamacoes ))}\SpecialCharTok{+}\FunctionTok{geom\_histogram}\NormalTok{(}\FunctionTok{aes}\NormalTok{(}\AttributeTok{y =} \FunctionTok{after\_stat}\NormalTok{(density)) , }\AttributeTok{binwidth=}\NormalTok{fd, }\AttributeTok{fill =} \StringTok{\textquotesingle{}lightblue\textquotesingle{}}\NormalTok{) }\SpecialCharTok{+} \FunctionTok{xlab}\NormalTok{(}\StringTok{\textquotesingle{}Total\_Reclamacoes \textquotesingle{}}\NormalTok{) }\SpecialCharTok{+} \FunctionTok{ylab}\NormalTok{(}\StringTok{\textquotesingle{}Densidade de Frequencia\textquotesingle{}}\NormalTok{) }\SpecialCharTok{+} \FunctionTok{labs}\NormalTok{(}\AttributeTok{title =} \StringTok{"Distribuicao dos dados Indice aproximada por Histograma"}\NormalTok{, }\AttributeTok{subtitle =} \StringTok{"Binarizacao pela Regra de FD"}\NormalTok{) }\SpecialCharTok{+} \FunctionTok{geom\_vline}\NormalTok{(}\AttributeTok{xintercept=}\FunctionTok{c}\NormalTok{(}\FunctionTok{median}\NormalTok{(Bancos}\SpecialCharTok{$}\NormalTok{Total\_Reclamacoes ), }\FunctionTok{mean}\NormalTok{(Bancos}\SpecialCharTok{$}\NormalTok{Total\_Reclamacoes ))) }\SpecialCharTok{+} \FunctionTok{annotate}\NormalTok{(}\StringTok{"text"}\NormalTok{, }\AttributeTok{x=}\FunctionTok{median}\NormalTok{(Bancos}\SpecialCharTok{$}\NormalTok{Total\_Reclamacoes ) }\SpecialCharTok{+} \FloatTok{0.3}\NormalTok{, }\AttributeTok{y=}\FloatTok{0.02}\NormalTok{, }\AttributeTok{label=}\StringTok{"Mediana"}\NormalTok{, }\AttributeTok{angle=}\DecValTok{90}\NormalTok{) }\SpecialCharTok{+} \FunctionTok{annotate}\NormalTok{(}\StringTok{"text"}\NormalTok{, }\AttributeTok{x=}\FunctionTok{mean}\NormalTok{(Bancos}\SpecialCharTok{$}\NormalTok{Total\_Reclamacoes ) }\SpecialCharTok{+} \FloatTok{0.3}\NormalTok{, }\AttributeTok{y=}\FloatTok{0.02}\NormalTok{, }\AttributeTok{label=}\StringTok{"Media"}\NormalTok{, }\AttributeTok{angle=}\DecValTok{90}\NormalTok{) }\SpecialCharTok{+} \FunctionTok{geom\_density}\NormalTok{(}\AttributeTok{linetype =} \DecValTok{2}\NormalTok{) }\SpecialCharTok{+} \FunctionTok{theme\_classic}\NormalTok{()}


\NormalTok{g2 }\OtherTok{\textless{}{-}}\NormalTok{ Bancos }\SpecialCharTok{\%\textgreater{}\%}\NormalTok{ dplyr}\SpecialCharTok{::}\FunctionTok{select}\NormalTok{(Total\_Reclamacoes ) }\SpecialCharTok{\%\textgreater{}\%} \FunctionTok{ggplot}\NormalTok{(}\FunctionTok{aes}\NormalTok{(}\AttributeTok{x=}\NormalTok{Total\_Reclamacoes ))}\SpecialCharTok{+}\FunctionTok{geom\_histogram}\NormalTok{(}\FunctionTok{aes}\NormalTok{(}\AttributeTok{y =} \FunctionTok{after\_stat}\NormalTok{(density)) , }\AttributeTok{binwidth=}\NormalTok{sr, }\AttributeTok{fill =} \StringTok{\textquotesingle{}lightblue\textquotesingle{}}\NormalTok{) }\SpecialCharTok{+} \FunctionTok{xlab}\NormalTok{(}\StringTok{\textquotesingle{}Total\_Reclamacoes \textquotesingle{}}\NormalTok{) }\SpecialCharTok{+} \FunctionTok{ylab}\NormalTok{(}\StringTok{\textquotesingle{}Densidade de Frequencia\textquotesingle{}}\NormalTok{) }\SpecialCharTok{+} \FunctionTok{labs}\NormalTok{(}\AttributeTok{title =} \StringTok{"Distribuicao dos dados de Indice aproximada por Histograma"}\NormalTok{, }\AttributeTok{subtitle =} \StringTok{"Binarizacao pela Regra de SR"}\NormalTok{) }\SpecialCharTok{+} \FunctionTok{geom\_vline}\NormalTok{(}\AttributeTok{xintercept=}\FunctionTok{c}\NormalTok{(}\FunctionTok{median}\NormalTok{(Bancos}\SpecialCharTok{$}\NormalTok{Total\_Reclamacoes ), }\FunctionTok{mean}\NormalTok{(Bancos}\SpecialCharTok{$}\NormalTok{Total\_Reclamacoes ))) }\SpecialCharTok{+} \FunctionTok{annotate}\NormalTok{(}\StringTok{"text"}\NormalTok{, }\AttributeTok{x=}\FunctionTok{median}\NormalTok{(Bancos}\SpecialCharTok{$}\NormalTok{Total\_Reclamacoes ) }\SpecialCharTok{+} \FloatTok{0.3}\NormalTok{, }\AttributeTok{y=}\FloatTok{0.02}\NormalTok{, }\AttributeTok{label=}\StringTok{"Mediana"}\NormalTok{, }\AttributeTok{angle=}\DecValTok{90}\NormalTok{) }\SpecialCharTok{+} \FunctionTok{annotate}\NormalTok{(}\StringTok{"text"}\NormalTok{, }\AttributeTok{x=}\FunctionTok{mean}\NormalTok{(Bancos}\SpecialCharTok{$}\NormalTok{Total\_Reclamacoes ) }\SpecialCharTok{+} \FloatTok{0.3}\NormalTok{, }\AttributeTok{y=}\FloatTok{0.02}\NormalTok{, }\AttributeTok{label=}\StringTok{"Media"}\NormalTok{, }\AttributeTok{angle=}\DecValTok{90}\NormalTok{) }\SpecialCharTok{+} \FunctionTok{geom\_density}\NormalTok{(}\AttributeTok{linetype =} \DecValTok{2}\NormalTok{) }\SpecialCharTok{+} \FunctionTok{theme\_classic}\NormalTok{()}

\NormalTok{g1 }\SpecialCharTok{+}\NormalTok{ g2}
\end{Highlighting}
\end{Shaded}

\includegraphics{Projeto2_files/figure-latex/unnamed-chunk-15-1.pdf}

\hypertarget{qq-plot-2}{%
\subsubsection{Qq Plot}\label{qq-plot-2}}

Grafico Qqplot

\begin{Shaded}
\begin{Highlighting}[]
\NormalTok{ggpubr}\SpecialCharTok{::}\FunctionTok{ggqqplot}\NormalTok{(Bancos}\SpecialCharTok{$}\NormalTok{Total\_Reclamacoes )  }\SpecialCharTok{+}
 \FunctionTok{labs}\NormalTok{(}\AttributeTok{title=}\StringTok{"QQ plot Variavel Total\_Reclamacoes "}\NormalTok{,}
       \AttributeTok{y =} \StringTok{"Indice"}\NormalTok{)}
\end{Highlighting}
\end{Shaded}

\includegraphics{Projeto2_files/figure-latex/unnamed-chunk-16-1.pdf}

\hypertarget{teste-shapiro-wilk-2}{%
\subsubsection{Teste Shapiro-Wilk}\label{teste-shapiro-wilk-2}}

\begin{Shaded}
\begin{Highlighting}[]
\NormalTok{rstatix}\SpecialCharTok{::}\FunctionTok{shapiro\_test}\NormalTok{(Bancos}\SpecialCharTok{$}\NormalTok{Total\_Reclamacoes)}
\end{Highlighting}
\end{Shaded}

\begin{verbatim}
## # A tibble: 1 x 3
##   variable                 statistic      p.value
##   <chr>                        <dbl>        <dbl>
## 1 Bancos$Total_Reclamacoes     0.671 0.0000000193
\end{verbatim}

\hypertarget{variavel-clientes}{%
\section{Variavel Clientes}\label{variavel-clientes}}

\hypertarget{distribuiuxe7uxe3o-nuxe3o-normal-2}{%
\subsection{Distribuição Não
Normal}\label{distribuiuxe7uxe3o-nuxe3o-normal-2}}

\hypertarget{histograma-3}{%
\subsubsection{Histograma}\label{histograma-3}}

\begin{Shaded}
\begin{Highlighting}[]
\NormalTok{g1 }\OtherTok{\textless{}{-}}\NormalTok{ Bancos }\SpecialCharTok{\%\textgreater{}\%}\NormalTok{ dplyr}\SpecialCharTok{::}\FunctionTok{select}\NormalTok{(Clientes) }\SpecialCharTok{\%\textgreater{}\%} \FunctionTok{ggplot}\NormalTok{(}\FunctionTok{aes}\NormalTok{(}\AttributeTok{x=}\NormalTok{Clientes))}\SpecialCharTok{+}\FunctionTok{geom\_histogram}\NormalTok{(}\FunctionTok{aes}\NormalTok{(}\AttributeTok{y =} \FunctionTok{after\_stat}\NormalTok{(density)) , }\AttributeTok{binwidth=}\NormalTok{fd, }\AttributeTok{fill =} \StringTok{\textquotesingle{}lightblue\textquotesingle{}}\NormalTok{) }\SpecialCharTok{+} \FunctionTok{xlab}\NormalTok{(}\StringTok{\textquotesingle{}Clientes\textquotesingle{}}\NormalTok{) }\SpecialCharTok{+} \FunctionTok{ylab}\NormalTok{(}\StringTok{\textquotesingle{}Densidade de Frequencia\textquotesingle{}}\NormalTok{) }\SpecialCharTok{+} \FunctionTok{labs}\NormalTok{(}\AttributeTok{title =} \StringTok{"Distribuicao dos dados Clientes aproximada por Histograma"}\NormalTok{, }\AttributeTok{subtitle =} \StringTok{"Binarizacao pela Regra de FD"}\NormalTok{)}
\CommentTok{\#+ geom\_vline(xintercept=c(median(Bancos$Clientes), mean(Bancos$Clientes))) + annotate("text", x=median(Bancos$Clientes) + 0.3, y=0.02, label="Mediana", angle=90) + annotate("text", x=mean(Bancos$Clientes) + 0.3, y=0.02, label="Media", angle=90) + geom\_density(linetype = 2) + theme\_classic()}


\NormalTok{g2 }\OtherTok{\textless{}{-}}\NormalTok{ Bancos }\SpecialCharTok{\%\textgreater{}\%}\NormalTok{ dplyr}\SpecialCharTok{::}\FunctionTok{select}\NormalTok{(Clientes) }\SpecialCharTok{\%\textgreater{}\%} \FunctionTok{ggplot}\NormalTok{(}\FunctionTok{aes}\NormalTok{(}\AttributeTok{x=}\NormalTok{Clientes))}\SpecialCharTok{+}\FunctionTok{geom\_histogram}\NormalTok{(}\FunctionTok{aes}\NormalTok{(}\AttributeTok{y =} \FunctionTok{after\_stat}\NormalTok{(density)) , }\AttributeTok{binwidth=}\NormalTok{sr, }\AttributeTok{fill =} \StringTok{\textquotesingle{}lightblue\textquotesingle{}}\NormalTok{) }\SpecialCharTok{+} \FunctionTok{xlab}\NormalTok{(}\StringTok{\textquotesingle{}Clientes\textquotesingle{}}\NormalTok{) }\SpecialCharTok{+} \FunctionTok{ylab}\NormalTok{(}\StringTok{\textquotesingle{}Densidade de Frequencia\textquotesingle{}}\NormalTok{) }\SpecialCharTok{+} \FunctionTok{labs}\NormalTok{(}\AttributeTok{title =} \StringTok{"Distribuicao dos dados de Clientes aproximada por Histograma"}\NormalTok{, }\AttributeTok{subtitle =} \StringTok{"Binarizacao pela Regra de SR"}\NormalTok{) }

\CommentTok{\#+ geom\_vline(xintercept=c(median(Bancos$Clientes), mean(Bancos$Clientes))) + annotate("text", x=median(Bancos$Clientes) , y=0.02, label="Mediana", angle=90) + annotate("text", x=mean(Bancos$Clientes) , y=0.02, label="Media", angle=90) + geom\_density(linetype = 2) + theme\_classic()}

\NormalTok{g1 }\SpecialCharTok{+}\NormalTok{ g2}
\end{Highlighting}
\end{Shaded}

\includegraphics{Projeto2_files/figure-latex/unnamed-chunk-18-1.pdf}

\hypertarget{qq-plot-3}{%
\subsubsection{Qq Plot}\label{qq-plot-3}}

Grafico Qqplot

\begin{Shaded}
\begin{Highlighting}[]
\NormalTok{ggpubr}\SpecialCharTok{::}\FunctionTok{ggqqplot}\NormalTok{(Bancos}\SpecialCharTok{$}\NormalTok{Clientes)  }\SpecialCharTok{+}
 \FunctionTok{labs}\NormalTok{(}\AttributeTok{title=}\StringTok{"QQ plot Variavel Clientes"}\NormalTok{,}
       \AttributeTok{y =} \StringTok{"Indice"}\NormalTok{)}
\end{Highlighting}
\end{Shaded}

\includegraphics{Projeto2_files/figure-latex/unnamed-chunk-19-1.pdf}

\hypertarget{teste-shapiro-wilk-3}{%
\subsubsection{Teste Shapiro-Wilk}\label{teste-shapiro-wilk-3}}

\begin{Shaded}
\begin{Highlighting}[]
\NormalTok{rstatix}\SpecialCharTok{::}\FunctionTok{shapiro\_test}\NormalTok{(Bancos}\SpecialCharTok{$}\NormalTok{Clientes)}
\end{Highlighting}
\end{Shaded}

\begin{verbatim}
## # A tibble: 1 x 3
##   variable        statistic      p.value
##   <chr>               <dbl>        <dbl>
## 1 Bancos$Clientes     0.676 0.0000000234
\end{verbatim}

\hypertarget{variaveis-razao}{%
\section{Variaveis Razao}\label{variaveis-razao}}

\hypertarget{distribuiuxe7uxe3o-normal-1}{%
\subsection{Distribuição Normal}\label{distribuiuxe7uxe3o-normal-1}}

\hypertarget{histogramas}{%
\subsubsection{Histogramas}\label{histogramas}}

\begin{Shaded}
\begin{Highlighting}[]
\NormalTok{g1 }\OtherTok{\textless{}{-}}\NormalTok{ Bancos }\SpecialCharTok{\%\textgreater{}\%}\NormalTok{ dplyr}\SpecialCharTok{::}\FunctionTok{select}\NormalTok{(Razao) }\SpecialCharTok{\%\textgreater{}\%} \FunctionTok{ggplot}\NormalTok{(}\FunctionTok{aes}\NormalTok{(}\AttributeTok{x=}\NormalTok{Razao))}\SpecialCharTok{+}\FunctionTok{geom\_histogram}\NormalTok{(}\FunctionTok{aes}\NormalTok{(}\AttributeTok{y =} \FunctionTok{after\_stat}\NormalTok{(density)) , }\AttributeTok{binwidth=}\NormalTok{fd, }\AttributeTok{fill =} \StringTok{\textquotesingle{}lightblue\textquotesingle{}}\NormalTok{) }\SpecialCharTok{+} \FunctionTok{xlab}\NormalTok{(}\StringTok{\textquotesingle{}Razao\textquotesingle{}}\NormalTok{) }\SpecialCharTok{+} \FunctionTok{ylab}\NormalTok{(}\StringTok{\textquotesingle{}Densidade de Frequencia\textquotesingle{}}\NormalTok{) }\SpecialCharTok{+} \FunctionTok{labs}\NormalTok{(}\AttributeTok{title =} \StringTok{"Distribuicao dos dados Razao aproximada por Histograma"}\NormalTok{, }\AttributeTok{subtitle =} \StringTok{"Binarizacao pela Regra de FD"}\NormalTok{) }\SpecialCharTok{+} \FunctionTok{geom\_vline}\NormalTok{(}\AttributeTok{xintercept=}\FunctionTok{c}\NormalTok{(}\FunctionTok{median}\NormalTok{(Bancos}\SpecialCharTok{$}\NormalTok{Razao), }\FunctionTok{mean}\NormalTok{(Bancos}\SpecialCharTok{$}\NormalTok{Razao))) }\SpecialCharTok{+} \FunctionTok{annotate}\NormalTok{(}\StringTok{"text"}\NormalTok{, }\AttributeTok{x=}\FunctionTok{median}\NormalTok{(Bancos}\SpecialCharTok{$}\NormalTok{Razao) }\SpecialCharTok{+} \FloatTok{0.3}\NormalTok{, }\AttributeTok{y=}\FloatTok{0.05}\NormalTok{, }\AttributeTok{label=}\StringTok{"Mediana"}\NormalTok{, }\AttributeTok{angle=}\DecValTok{90}\NormalTok{) }\SpecialCharTok{+} \FunctionTok{annotate}\NormalTok{(}\StringTok{"text"}\NormalTok{, }\AttributeTok{x=}\FunctionTok{mean}\NormalTok{(Bancos}\SpecialCharTok{$}\NormalTok{Razao) }\SpecialCharTok{+} \FloatTok{0.3}\NormalTok{, }\AttributeTok{y=}\FloatTok{0.05}\NormalTok{, }\AttributeTok{label=}\StringTok{"Media"}\NormalTok{, }\AttributeTok{angle=}\DecValTok{90}\NormalTok{) }\SpecialCharTok{+} \FunctionTok{geom\_density}\NormalTok{(}\AttributeTok{linetype =} \DecValTok{2}\NormalTok{) }\SpecialCharTok{+} \FunctionTok{theme\_classic}\NormalTok{()}


\NormalTok{g2 }\OtherTok{\textless{}{-}}\NormalTok{ Bancos }\SpecialCharTok{\%\textgreater{}\%}\NormalTok{ dplyr}\SpecialCharTok{::}\FunctionTok{select}\NormalTok{(Razao) }\SpecialCharTok{\%\textgreater{}\%} \FunctionTok{ggplot}\NormalTok{(}\FunctionTok{aes}\NormalTok{(}\AttributeTok{x=}\NormalTok{Razao))}\SpecialCharTok{+}\FunctionTok{geom\_histogram}\NormalTok{(}\FunctionTok{aes}\NormalTok{(}\AttributeTok{y =} \FunctionTok{after\_stat}\NormalTok{(density)) , }\AttributeTok{binwidth=}\NormalTok{sr, }\AttributeTok{fill =} \StringTok{\textquotesingle{}lightblue\textquotesingle{}}\NormalTok{) }\SpecialCharTok{+} \FunctionTok{xlab}\NormalTok{(}\StringTok{\textquotesingle{}Razao\textquotesingle{}}\NormalTok{) }\SpecialCharTok{+} \FunctionTok{ylab}\NormalTok{(}\StringTok{\textquotesingle{}Densidade de Frequencia\textquotesingle{}}\NormalTok{) }\SpecialCharTok{+} \FunctionTok{labs}\NormalTok{(}\AttributeTok{title =} \StringTok{"Distribuicao dos dados de Razao aproximada por Histograma"}\NormalTok{, }\AttributeTok{subtitle =} \StringTok{"Binarizacao pela Regra de SR"}\NormalTok{) }\SpecialCharTok{+} \FunctionTok{geom\_vline}\NormalTok{(}\AttributeTok{xintercept=}\FunctionTok{c}\NormalTok{(}\FunctionTok{median}\NormalTok{(Bancos}\SpecialCharTok{$}\NormalTok{Razao), }\FunctionTok{mean}\NormalTok{(Bancos}\SpecialCharTok{$}\NormalTok{Razao))) }\SpecialCharTok{+} \FunctionTok{annotate}\NormalTok{(}\StringTok{"text"}\NormalTok{, }\AttributeTok{x=}\FunctionTok{median}\NormalTok{(Bancos}\SpecialCharTok{$}\NormalTok{Razao) }\SpecialCharTok{+} \FloatTok{0.3}\NormalTok{, }\AttributeTok{y=}\FloatTok{0.05}\NormalTok{, }\AttributeTok{label=}\StringTok{"Mediana"}\NormalTok{, }\AttributeTok{angle=}\DecValTok{90}\NormalTok{) }\SpecialCharTok{+} \FunctionTok{annotate}\NormalTok{(}\StringTok{"text"}\NormalTok{, }\AttributeTok{x=}\FunctionTok{mean}\NormalTok{(Bancos}\SpecialCharTok{$}\NormalTok{Razao) }\SpecialCharTok{+} \FloatTok{0.3}\NormalTok{, }\AttributeTok{y=}\FloatTok{0.05}\NormalTok{, }\AttributeTok{label=}\StringTok{"Media"}\NormalTok{, }\AttributeTok{angle=}\DecValTok{90}\NormalTok{) }\SpecialCharTok{+} \FunctionTok{geom\_density}\NormalTok{(}\AttributeTok{linetype =} \DecValTok{2}\NormalTok{) }\SpecialCharTok{+} \FunctionTok{theme\_classic}\NormalTok{()}

\NormalTok{g1 }\SpecialCharTok{+}\NormalTok{ g2}
\end{Highlighting}
\end{Shaded}

\includegraphics{Projeto2_files/figure-latex/unnamed-chunk-21-1.pdf}

\hypertarget{qqplot}{%
\subsubsection{QQPLOT}\label{qqplot}}

Grafico Qqplot

\begin{Shaded}
\begin{Highlighting}[]
\NormalTok{ggpubr}\SpecialCharTok{::}\FunctionTok{ggqqplot}\NormalTok{(Bancos}\SpecialCharTok{$}\NormalTok{Razao)}
\end{Highlighting}
\end{Shaded}

\includegraphics{Projeto2_files/figure-latex/unnamed-chunk-22-1.pdf}

\hypertarget{shapiro-wilk-normality-test}{%
\subsubsection{Shapiro-Wilk Normality
Test}\label{shapiro-wilk-normality-test}}

\begin{Shaded}
\begin{Highlighting}[]
\NormalTok{rstatix}\SpecialCharTok{::}\FunctionTok{shapiro\_test}\NormalTok{(Bancos}\SpecialCharTok{$}\NormalTok{Razao)}
\end{Highlighting}
\end{Shaded}

\begin{verbatim}
## # A tibble: 1 x 3
##   variable     statistic p.value
##   <chr>            <dbl>   <dbl>
## 1 Bancos$Razao     0.991   0.977
\end{verbatim}

\hypertarget{coeficiente-de-correlauxe7uxe3o}{%
\section{Coeficiente de
Correlação}\label{coeficiente-de-correlauxe7uxe3o}}

--

Na estatística o coeficiente de correlação de Pearson (r) mede a relação
que existe entre duas variáveis dentro de uma mesma escala métrica.

A função do coeficiente de correlação é determinar qual é a intensidade
da relação que existe entre conjuntos de dados ou informações
conhecidas.

O valor do coeficiente de correlação pode variar entre -1 e 1. Para
interpretar o coeficiente é preciso saber que 1 significa que a
correlação entre as variáveis é perfeita positiva e -1 significa que é
perfeita negativa. Se o coeficiente for igual a 0 significa que as
variáveis não dependem uma da outra.

\hypertarget{correlograma}{%
\subsubsection{Correlograma}\label{correlograma}}

--

Utilizamos a função corrplot() para criar uma visualização grafica da
matrix de correlação.

\begin{Shaded}
\begin{Highlighting}[]
\NormalTok{corrplot }\SpecialCharTok{::} \FunctionTok{corrplot}\NormalTok{(}\FunctionTok{cor}\NormalTok{(Bancos[, }\SpecialCharTok{{-}}\DecValTok{1}\NormalTok{]), }\AttributeTok{method=}\StringTok{\textquotesingle{}number\textquotesingle{}}\NormalTok{, , }\AttributeTok{tl.col =} \StringTok{\textquotesingle{}black\textquotesingle{}}\NormalTok{, }\AttributeTok{cl.ratio =} \FloatTok{0.2}\NormalTok{, }\AttributeTok{tl.srt =} \DecValTok{45}\NormalTok{, }\AttributeTok{col =} \FunctionTok{COL2}\NormalTok{(}\StringTok{\textquotesingle{}RdYlBu\textquotesingle{}}\NormalTok{, }\DecValTok{15}\NormalTok{))}
\end{Highlighting}
\end{Shaded}

\includegraphics{Projeto2_files/figure-latex/unnamed-chunk-24-1.pdf}

As maiores coorelação são entre o total de quantidade de clientes,
reclamações recebidas e reclamações procedentes.

\hypertarget{scatterplot-matrix}{%
\section{Scatterplot Matrix}\label{scatterplot-matrix}}

A função ggpairs() do pacote GGally nos permite construir uma ótima
matriz de gráfico de dispersão. Gráficos de dispersão de cada par
visualizados no lado esquerdo do gráfico e valor de correlação de
Pearson e significância exibidos no lado direito.

\begin{Shaded}
\begin{Highlighting}[]
\CommentTok{\#create }
\FunctionTok{ggpairs}\NormalTok{(Bancos[, }\SpecialCharTok{{-}}\DecValTok{1}\NormalTok{])}
\end{Highlighting}
\end{Shaded}

\hypertarget{regressuxe3o-linear}{%
\section{Regressão linear}\label{regressuxe3o-linear}}

A correlação é uma relação entre duas variáveis, sendo a primeira
dependente e a segunda independente. A análise da correlação tem o
objetivo de atestar se existe uma relação de dependência significativa
entre duas variáveis, é feito através do coeficiente de correlação,
variando de -- 1 a 1, com -1 correlação linear negativa, 0 nenhuma
correlação e 1 correlação linear positiva.

Um exemplo em que se pode utilizar o princípio da correlação, um gerente
de recursos humanos deseja saber se existe ligação entre o número de
horas de treinamento de um funcionário e o número de reclamações que o
funcionário recebeu. Caso haja correlação linear positiva, quer dizer
que quanto mais horas de treinamento se gasta com funcionários, menos
reclamações. Nesse caso, a variável independente é a quantidade de horas
de treinamento para o funcionário, e a variável dependente é o número de
reclamações.

A análise de regressão linear pode ser feita após a garantia de
correlação linear entre duas variáveis, busca encontrar a reta que
melhor modela, compensando-se os resíduos, a correlação entre as
variáveis envolvidas, e essa reta pode ser utilizada para fazer
previsões de valores da variável dependente a partir de valores da
variável independente. O mesmo exemplo sobre o treinamento de
funcionários e número de reclamações valem também para a regressão
linear, a ``saída'' da regressão linear seria uma função que expressaria
da melhor maneira a relação entre os valores das variáveis envolvidas.

Para concluir, a análise de correlação junto com a regressão linear têm
por objetivo maior inferir se há relação entre duas variáveis com base
em um conjunto de dados anterior, e com base nessa inferência, caso
positiva para que haja de fato relação, seja possível obter uma função
que relaciona uma variável a outra para que se possa prever valores
futuros para a variável dependente com base em valores da variável
independente. A diferença entre a correlação e regressão linear é que a
2ª obtém uma função da relação que melhor a expressa, enquanto que a 1ª
indica se há relação.

Regressão linear é o método com o qual se encontra a reta que melhor
descreve a relação entre os dados. A reta é criada com uma equação
linear y = b + m1 * x1\ldots mn*xn, onde b será o coeficiente linear, e
m o coeficiente angular, sendo um para cada variável preditora existente
nos dados.

A função que realiza o ajuste da reta ou modelo de regressão linear no R
é a lm().

O comando summary() poderá indicar se os seus parâmetros estimados são
significativos ou não, ou seja, se é possível assumir que são diferentes
de zero.

\begin{Shaded}
\begin{Highlighting}[]
\CommentTok{\#Simples, variavel explicativa idade}
\NormalTok{modelo}\FloatTok{.1} \OtherTok{\textless{}{-}} \FunctionTok{lm}\NormalTok{(Procedentes }\SpecialCharTok{\textasciitilde{}}\NormalTok{ Clientes, }\AttributeTok{data =}\NormalTok{ Bancos)}
\FunctionTok{summary}\NormalTok{(modelo}\FloatTok{.1}\NormalTok{, }\AttributeTok{type =} \StringTok{\textquotesingle{}html\textquotesingle{}}\NormalTok{)}
\end{Highlighting}
\end{Shaded}

Call: lm(formula = Procedentes \textasciitilde{} Clientes, data =
Bancos)

Residuals: Min 1Q Median 3Q Max -571.71 -151.63 -77.27 76.13 1202.91

Coefficients: Estimate Std. Error t value
Pr(\textgreater\textbar t\textbar)\\
(Intercept) 1.276e+02 5.832e+01 2.188 0.0346 *\\
Clientes 1.334e-05 1.484e-06 8.986 3.84e-11 *** --- Signif. codes: 0
`\emph{\textbf{' 0.001 '}' 0.01 '}' 0.05 `.' 0.1 ' ' 1

Residual standard error: 314.4 on 40 degrees of freedom Multiple
R-squared: 0.6687, Adjusted R-squared: 0.6604 F-statistic: 80.74 on 1
and 40 DF, p-value: 3.842e-11

\begin{Shaded}
\begin{Highlighting}[]
\NormalTok{modelo}\FloatTok{.1}\SpecialCharTok{$}\NormalTok{coefficients}
\end{Highlighting}
\end{Shaded}

\begin{verbatim}
##  (Intercept)     Clientes 
## 1.275916e+02 1.333451e-05
\end{verbatim}

E temos a equação da reta ajustada: E(Y ) = 1.275916e+02 +1.333451e-05 ∗
Clientes

Essa função também retorna a medida R²(adjusted R-squared), que indica o
quanto da variação presente nos dados está sendo explicada pela
covariável.

Quanto maior o R², mais explicativo é o modelo linear, ou seja, melhor
ele se ajusta à amostra. Com R² = 0.6687 significa que o modelo linear
explica 66,87\% da variância da variável dependente a partir do
regressores (variáveis independentes) incluídas naquele modelo linear.

\begin{Shaded}
\begin{Highlighting}[]
\FunctionTok{ggplot}\NormalTok{(Bancos, }\AttributeTok{mapping =} \FunctionTok{aes}\NormalTok{(Procedentes, Clientes)) }\SpecialCharTok{+}
   \FunctionTok{geom\_point}\NormalTok{() }\SpecialCharTok{+}
   \FunctionTok{geom\_smooth}\NormalTok{(}\AttributeTok{method =} \StringTok{"lm"}\NormalTok{)}
\end{Highlighting}
\end{Shaded}

\begin{verbatim}
## `geom_smooth()` using formula = 'y ~ x'
\end{verbatim}

\includegraphics{Projeto2_files/figure-latex/unnamed-chunk-27-1.pdf}

\hypertarget{section}{%
\subsection{}\label{section}}

\hypertarget{carregando-o-banco-de-dados-1}{%
\section{Carregando o Banco de
Dados}\label{carregando-o-banco-de-dados-1}}

--

Utilizamos o github para download do arquivo baixado do Banco Central e
a função read\_excel() para criar o Banco de dados Bancos3 com os dados
do arquivo Bancos4.xlsx.

\begin{Shaded}
\begin{Highlighting}[]
\FunctionTok{download.file}\NormalTok{( }\AttributeTok{url =} \StringTok{"https://github.com/taianasantos/analise/blob/main/Bancosconna.xlsx?raw=true"}\NormalTok{, }\AttributeTok{quiet =} \ConstantTok{TRUE}\NormalTok{,  }\AttributeTok{mode =} \StringTok{"wb"}\NormalTok{, }\AttributeTok{destfile =} \StringTok{"Bancos1.xlsx"}\NormalTok{ )}

\NormalTok{Bancos }\OtherTok{\textless{}{-}}\NormalTok{ readxl }\SpecialCharTok{::} \FunctionTok{read\_excel}\NormalTok{(}\StringTok{"Bancos1.xlsx"}\NormalTok{)}
\end{Highlighting}
\end{Shaded}

\hypertarget{dados-faltantes}{%
\section{Dados Faltantes}\label{dados-faltantes}}

Completude (SUBSTANTIVO) - qualidade, estado ou propriedade do que é
completo, perfeito, acabado.

We should be suspicious of any dataset (large or small) which appears
perfect. Tradução (``Devemos suspeitar de qualquer conjunto de dados
(grande ou pequeno) que pareça perfeito.'') --- David J. Hand

A definição de Completude é a qualidade, estado ou propriedade do que é
completo, perfeito, acabado. No mundo ideal um conjunto de dados seria
totalmente preenchido, mas na pratica é muito comum que ele não esteja
totalmente preenchido, terá em algumas colunas ou linhas, valores
chamados ``NaN'' (do inglês ``Not a Number''), que são os valores
faltantes.

Algumas possibilidades desses dados faltantes podem ser: perguntas não
respondidas em uma entrevista, dados perdidos por coleta equivocada de
registro, muitas vezes, quem estava coletando o conjunto de dados não
encontrou a informação necessária; ou ainda, no momento de se transferir
os dados de um lugar para outro, a informação foi perdida; enfim, há
vários motivos que podem fazer com que um dado falte em uma base de
dados.

Algumas Informações importantes são perdidas, correlações ficam
distorcidas ou pode não ter um número satisfatório de pessoas para
generalizar a situação para uma população maior.

O impacto desses dados faltantes na inferência estatística é
potencialmente alto, ainda mais em alguns casos onde o público que
possui dado ausente é sistematicamente diferente daquele que os dados
estão completos. O simplesmente descartá-los pode levar a resultados
enviesados.

\hypertarget{tipos-de-dados-faltantes}{%
\subsection{Tipos de dados faltantes?}\label{tipos-de-dados-faltantes}}

\begin{enumerate}
\def\labelenumi{\arabic{enumi}.}
\tightlist
\item
  Faltantes completamente aleatórios (missingnes completley at random).
  A probabilidade de estarem ausentes é independente de qualquer
  observação no banco de dados.
\item
  Faltantes aleatórios (missingnes at random). Os dados faltantes não
  respeitam uma distribuição aleatória como as amostras com dados
  observados.
\item
  Faltantes não aleatórios (missing not at random). A probabilidade é
  que os valores faltantes dependam não só dos dados observados assim
  como dos dados não observados.
\end{enumerate}

É importante identificar o tipo correto de dados faltantes para não
introduzir um viés aos dados e utilizar a melhor técninca para tratar os
dados ausentes.

Algumas técnicas de tratamenos dos ``NaN'':

\begin{enumerate}
\def\labelenumi{\arabic{enumi}.}
\item
  Deletar a variável/atributo que possui dados faltantes.
\item
  Deletar somente as amostras com dados faltantes.
\item
  Preencher (fill in) os dados faltantes por alguma estatística (média,
  mediana, etc)
\item
  Modelar o dado faltante
\end{enumerate}

\begin{Shaded}
\begin{Highlighting}[]
\NormalTok{Bancos }\SpecialCharTok{\%\textgreater{}\%}\NormalTok{ dlookr}\SpecialCharTok{::}\FunctionTok{diagnose}\NormalTok{()}
\end{Highlighting}
\end{Shaded}

\begin{verbatim}
## # A tibble: 6 x 6
##   variables         types     missing_count missing_percent unique_count uniqu~1
##   <chr>             <chr>             <int>           <dbl>        <int>   <dbl>
## 1 Instituicao       character             0             0            154   1    
## 2 Indice            numeric             112            72.7           43   0.279
## 3 Procedentes       numeric               0             0             65   0.422
## 4 Total_Reclamacoes numeric               0             0             86   0.558
## 5 Clientes          numeric               0             0            144   0.935
## 6 Razao             numeric               0             0            102   0.662
## # ... with abbreviated variable name 1: unique_rate
\end{verbatim}

\begin{Shaded}
\begin{Highlighting}[]
\FunctionTok{sum}\NormalTok{(}\FunctionTok{is.na}\NormalTok{(Bancos)) }
\end{Highlighting}
\end{Shaded}

\begin{verbatim}
## [1] 112
\end{verbatim}

\hypertarget{imputando-dados-com-mice}{%
\subsection{Imputando dados com mice}\label{imputando-dados-com-mice}}

Métodos multivariados para tratamento de missing values A função mice()
é utilizada para construir o modelo, que produz várias imputações
(valores de substituição) para os dados ausentes, sendo o método baseado
na especificação condicional, em que cada variável incompleta é imputada
por um modelo separado.

Dados faltantes na Variavel Indice, quando a quantidade de reclamações
procedentes é menor que 30, fantantes não aleatórios.

A função mice possui vários métodos, como por exemplo:

\begin{Shaded}
\begin{Highlighting}[]
\CommentTok{\# Realiza a imputacao mice}
\NormalTok{dadosmice }\OtherTok{\textless{}{-}} \FunctionTok{mice}\NormalTok{(}
  \AttributeTok{data =}\NormalTok{ Bancos[ ,}\SpecialCharTok{{-}}\DecValTok{1}\NormalTok{],}
  \AttributeTok{meth =} \StringTok{"rf"}
\NormalTok{) }\SpecialCharTok{\%\textgreater{}\%} 
  \FunctionTok{complete}\NormalTok{() }
\end{Highlighting}
\end{Shaded}

\begin{verbatim}
## 
##  iter imp variable
##   1   1  Indice
##   1   2  Indice
##   1   3  Indice
##   1   4  Indice
##   1   5  Indice
##   2   1  Indice
##   2   2  Indice
##   2   3  Indice
##   2   4  Indice
##   2   5  Indice
##   3   1  Indice
##   3   2  Indice
##   3   3  Indice
##   3   4  Indice
##   3   5  Indice
##   4   1  Indice
##   4   2  Indice
##   4   3  Indice
##   4   4  Indice
##   4   5  Indice
##   5   1  Indice
##   5   2  Indice
##   5   3  Indice
##   5   4  Indice
##   5   5  Indice
\end{verbatim}

\begin{Shaded}
\begin{Highlighting}[]
\CommentTok{\#skim(dadosmice)}
\end{Highlighting}
\end{Shaded}

\begin{Shaded}
\begin{Highlighting}[]
\CommentTok{\# Realiza a imputacao mice}
\NormalTok{dadosmice2 }\OtherTok{\textless{}{-}} \FunctionTok{mice}\NormalTok{(}
  \AttributeTok{data =}\NormalTok{ Bancos[ ,}\SpecialCharTok{{-}}\DecValTok{1}\NormalTok{],}
  \AttributeTok{meth =} \StringTok{"cart"}
\NormalTok{) }\SpecialCharTok{\%\textgreater{}\%} 
  \FunctionTok{complete}\NormalTok{() }
\end{Highlighting}
\end{Shaded}

\begin{verbatim}
## 
##  iter imp variable
##   1   1  Indice
##   1   2  Indice
##   1   3  Indice
##   1   4  Indice
##   1   5  Indice
##   2   1  Indice
##   2   2  Indice
##   2   3  Indice
##   2   4  Indice
##   2   5  Indice
##   3   1  Indice
##   3   2  Indice
##   3   3  Indice
##   3   4  Indice
##   3   5  Indice
##   4   1  Indice
##   4   2  Indice
##   4   3  Indice
##   4   4  Indice
##   4   5  Indice
##   5   1  Indice
##   5   2  Indice
##   5   3  Indice
##   5   4  Indice
##   5   5  Indice
\end{verbatim}

\begin{Shaded}
\begin{Highlighting}[]
\CommentTok{\#skim(dadosmice2)}
\end{Highlighting}
\end{Shaded}

\hypertarget{dashboard-shiny}{%
\section{Dashboard Shiny}\label{dashboard-shiny}}

\hypertarget{section-1}{%
\section{}\label{section-1}}

\includegraphics{https://raw.githubusercontent.com/taianasantos/analise/main/shyne1.jpg}

\includegraphics{https://raw.githubusercontent.com/taianasantos/analise/main/shyne2.png}

\hypertarget{conclusuxf5es}{%
\section{Conclusões}\label{conclusuxf5es}}

--

O Ranking de reclamações é muito importante para dar transparencia e
confiança dos serviços prestados pelas Instituições Financeiras.
Percebemos que Instituições recebem muitas reclamações improcedentes,
cerca de 64\%, esse numero elevado pode ser fruto de insatisfação com o
produto contratado, falta de conhecimento ou pouca transparência das
normas bancárias.

Algumas sugestões propostas para uma redução das reclamações recebidas :

Melhoria dos canais de ouvidoria para reduzir a quantidade de
reclamações recebidas pelo Banco Central.

Transparência nas informações prestadas aos clientes.

Investimento em Treinamento e ampliação das equipes de atendimento.

Aperfeiçoamento dos canais de atendimento.

Linguagem simples para melhoria da compreensão dos clientes.

\hypertarget{bibliografia}{%
\section{Bibliografia}\label{bibliografia}}

\url{https://didatica.tech/regressao-linear-com-linguagem-r/}

\url{https://didatica.tech/o-que-sao-dados-missing/}

\url{https://www.ufrgs.br/wiki-r/index.php?title=Tratamento_de_Missing_Values\#Tratamento_de_missing_values_com_Knn_no_R}

\url{https://www.statology.org/scatterplot-matrix-in-r/}

\url{https://stefvanbuuren.name/fimd/sec-modelform.html}

\end{document}
